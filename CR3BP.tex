\documentclass{article}
\usepackage{amsmath}
\usepackage{tikz}
\usepackage{tikz-3dplot}
% \usepackage{esvect}
\usepackage{cancel}
\usepackage{float}
\usepackage{comment}
\usepackage[a4paper, left=3.5cm,right=3.5cm, top=3cm, bottom=3cm]{geometry}

\newcommand{\dd}[2]{\frac{\mathrm{d}#1}{\mathrm{d}#2}}
\newcommand{\vv}[1]{\boldsymbol{#1}}
\newcommand{\vh}[1]{\boldsymbol{\hat{#1}}}

\pagestyle{empty}
\begin{document}

% \section*{Diagrams}
The circular restricted three body problem (CR3BP) is a special case of the three body problem. In the CR3BP (much like in Keplerian 2-body dynamics), we neglect the mass of satellite $S$, while treating the larger celestial body $m_1$ and smaller celestial body $m_2$ as point masses. Crucially, these bodies must orbit one another in circular orbits. In other words, they both orbit about their inertially fixed barycenter $c$ \textit{at constant velocity and distance}.

\begin{figure}[H]
    \centering
    \tdplotsetmaincoords{70}{30}
    \begin{tikzpicture}[
        planet/.style = {circle, draw=black, ball color=#1, node contents={}, inner sep = 0.1cm}, 
        point/.style = {circle, draw=#1, fill=#1, node contents={}, inner sep=0.05cm},
        >=latex,
        tdplot_main_coords
        ]

        \draw[dotted, fill=none, red](0,0,0) circle (1);
        \draw[dotted, fill=none, blue](0,0,0) circle (3);
        \node (B1) at (-1,0,0)     [planet=red, label=left:$m_1$];
        \node (B2) at (3,0,0)      [planet=blue, label=right:$m_2$];

        \draw[dashed, red] (B1) -- (0,0,0) node[midway] {$x_1$};
        \draw[dashed, blue] (0,0,0) -- (B2) node[midway] {$x_2$};
        
        \node (o) at (0,0,0)      [point=black, label=below:$c$];
        \node (r) at (1.25,2,1.25)    [point=purple, label=left:$S$];
        \draw[dashed, purple] (0,0,0) -- (0, 2, 0) node[midway, right]  {$y$} -- (1.25, 2, 0) node[midway, above]  {$x$} -- (r) node[midway, right] {$z$};

        \draw[->, gray] (0,0,0) -- (0.75,0,0) node[near end, below right] {$\vh{x}$};
        \draw[->, gray] (0,0,0) -- (0,0.75,0) node[near end, above] {$\vh{y}$};
        \draw[->, gray] (0,0,0) -- (0,0,0.75) node[near end, above left] {$\vh{z}$};
    \end{tikzpicture}

    \caption{Geometry. The purple satellite is located by $\vv{r}=x\vh{x}+y\vh{y}+z\vh{z}$, and the spheres are the celestial bodies. Note that $\vv{r}$ is not constrained to the $xy$ plane, but $m_1$ and $m_2$ definitionally are. The bodies are always along the $x$ axis, as the $c$ frame rotates with them}
\end{figure}

Because $\vh{x}$ points from $c$ to $m_2$, which is not inertially stationary, the $xyz$ frame is rotating. Specifically, it is rotating positively about $z$. Because the celestial bodies are in a circular orbit, their rates of rotation about $c$ are constant. This means that the $xyz$ frame rotates at a constant rate of $\vv{\omega}=\Omega\vh{z}$ 

We can also find a relationship between $x_1$, $x_2$, $r_{12}$, $m_1$, and $m_2$. Because $c$ is the barycenter,
\[x_1m_1=x_2m_2\]
or
\[\frac{x_1}{m_2}=\frac{x_2}{m_1}\]
We can use this to find that
\[\begin{aligned}
\frac{x_1}{x_1+x_2}&=\frac{x_1/m_2}{x_1/m_2+x_2/m_2}\\
&=\frac{x_2/m_1}{x_2/m_1+x_2/m_2}\\
&=\frac{1/m_1}{1/m_1+1/m_2}\\
&=\frac{m_2}{m_2+m_1}\\
\end{aligned}\]

Defining $r_{12}=x_1+x_2$ as the distance between the two celestial bodies, and $M=m_1+m_2$ as their total mass, we get that
\[\boxed{\frac{x_1}{r_{12}}=\frac{m_2}{M}}\]
and similarly
\[\boxed{\frac{x_2}{r_{12}}=\frac{m_1}{M}}\]

Which can be rewritten as
\[\boxed{\frac{x_1}{m_2}=\frac{x_2}{m_1}=\frac{r_{12}}{M}}\]

\section*{Kinematics}
The transport theorem states that the inertial (fixed $f$ frame) derivative of a vector $u$ (expressed in the rotating $c$ frame) is

\[\dd{^f \vv{u}}{t}=\dd{^c\vv{u}}{t}+\vv{\omega}^{cf}\times\vv{u}\]

Where $\dd{^f}{t}$ denotes the derivative in the coordinates of the fixed frame $f$, and $\dd{^c}{t}$ denotes derivative in the coordinates of the rotating frame $c$, and $\vv{\omega}^{cf}$ denotes the angular velocity of $c$ in $f$. For this case, $\vv{\omega}^{cf}=\Omega\vh{z}$. We can find $\Omega$

\begin{comment}
Now we will solve for $\Omega$. Note that, because $c$ is the center of mass of $m_1$ and $m_2$, 

\[\begin{aligned}
    \vv{F_{\text{on2}}}&=-\frac{Gm_1m_2}{r_{12}^2}\vh{x}\\
    \cancel{m_2}\vv{a}_2^f&=-\frac{Gm_1\cancel{m_2}}{r_{12}^2}\vh{x}\\
    \dd{^f}{t}\left(\dd{^f}{t}x_2\vh{x}\right)&=-\frac{Gm_1}{r_{12}^2}\vh{x}\\
    \dd{^f}{t}\left(\cancelto{0}{\dd{^c x_2\vh{x}}{t}}+\Omega\vh{z}\times x_2\vh{x}\right)&=-\frac{Gm_1}{r_{12}^2}\vh{x}\\
    \dd{^f}{t}\Omega x_2\vh{y}&=-\frac{Gm_1}{r_{12}^2}\vh{x}\\
    \cancelto{0}{\dd{^c}{t}\Omega x_2 \vh{y}}+\Omega\vh{z}\times\Omega x_2\vh{y}&=-\frac{Gm_1}{r_{12}^2}\vh{x}\\
    -\Omega^2 x_2\vh{x}&=-\frac{Gm_1}{r_{12}^2}\vh{x}\\
    \Omega^2x_2&=\frac{Gm_1}{r_{12}^2}\\
    \Omega&=\sqrt{\frac{G}{r_{12}^2}\frac{m_1}{x_2}}\\
    \Omega&=\sqrt{\frac{GM}{r_{12}^3}}\\
\end{aligned}\]

Defining $\mu$ conventionally as $\mu=GM$,
\[\boxed{\Omega=\sqrt{\frac{\mu}{r_{12}^3}}}\]

\end{comment}

For the satellite's position in the CR3BP frame $\vv{r}=x\vh{x}+y\vh{y}+z\vh{z}$, we will find the inertial acceleration to generate equations of motion.

\[
\begin{aligned}
    \vv{\dot{r}}&=\dd{^c\vv{r}}{t}+\vv{\omega}^{cf}\times\vv{r}\\
    &=\dd{^c}{t}\left(x\vh{x}+y\vh{y}+z\vh{z}\right)+\left(\Omega\vh{z}\times\left(x\vh{x}+y\vh{y}+z\vh{z}\right)\right)\\
    &=\left(\dot{x}\vh{x}+\dot{y}\vh{y}+\dot{z}\vh{z}\right)+\left(\Omega x\vh{y}-\Omega y\vh{x}\right)\\
    &=\left(\dot{x}-\Omega y\right)\vh{x}+\left(\dot{y}+\Omega x\right)\vh{y}+\dot{z}\vh{z}
\end{aligned}
\]

\[
\begin{aligned}
    \vv{\ddot{r}}&=\dot{\vv{r}}\\
    &=\dd{^c}{t}\left(\left(\dot{x}-\Omega y\right)\vh{x}+\left(\dot{y}+\Omega x\right)\vh{y}+\dot{z}\vh{z}\right)\\
        &\phantom{=}\quad+\Omega \vh{z}\times\left(\left(\dot{x}-\Omega y\right)\vh{x}+\left(\dot{y}+\Omega x\right)\vh{y}+\dot{z}\vh{z}\right)\\
    &=\left(\ddot{x}-\Omega \dot{y}\right)\vh{x}+\left(\ddot{y}+\Omega \dot{x}\right)\vh{y}+\ddot{z}\vh{z}\\
        &\phantom{=}\quad+\left(\left(\Omega\dot{x}-\Omega^2 y\right)\vh{y}-\left(\Omega\dot{y}+\Omega^2 x\right)\vh{x}\right)\\
    &=\left(\ddot{x}-2\Omega\dot{y}-\Omega^2 x\right)\vh{x} + \left(\ddot{y}+2\Omega\dot{x}-\Omega^2 y\right)\vh{y}+\ddot{z}\vh{z}\\
\end{aligned}
\]

\[\boxed{\vv{\ddot{r}}=\left(\ddot{x}-2\Omega\dot{y}-\Omega^2 x\right)\vh{x} + \left(\ddot{y}+2\Omega\dot{x}-\Omega^2 y\right)\vh{y}+\ddot{z}\vh{z}}\]




\end{document}