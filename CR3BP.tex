\documentclass{article}
\usepackage{amsmath}
\usepackage{tikz}
\usepackage{tikz-3dplot}
% \usepackage{esvect}
\usepackage{cancel}
\usepackage{float}
\usepackage{comment}
\usepackage[a4paper, left=3.5cm,right=3.5cm, top=3cm, bottom=3cm]{geometry}

\newcommand{\dd}[2]{\frac{\mathrm{d}#1}{\mathrm{d}#2}}
\newcommand{\vv}[1]{\boldsymbol{#1}}
\newcommand{\vh}[1]{\boldsymbol{\hat{#1}}}

\pagestyle{empty}
\begin{document}

% \section*{Diagrams}
The circular restricted three body problem (CR3BP) is a special case of the three body problem. In the CR3BP (much like in Keplerian 2-body dynamics), we neglect the mass of satellite $S$, while treating the larger celestial body $m_1$ and smaller celestial body $m_2$ as point masses. Crucially, these bodies must orbit one another in circular orbits. In other words, they both orbit about their inertially fixed barycenter $c$ \textit{at constant velocity and distance}.

\begin{figure}[H]
    \centering
    \tdplotsetmaincoords{70}{30}
    \begin{tikzpicture}[
        planet/.style = {circle, draw=black, ball color=#1, node contents={}, inner sep = 0.1cm}, 
        point/.style = {circle, draw=#1, fill=#1, node contents={}, inner sep=0.05cm},
        >=latex,
        tdplot_main_coords
        ]

        \node (B1) at (-1,0,0)     [planet=red, label=left:$m_1$];
        \node (B2) at (3,0,0)      [planet=blue, label=right:$m_2$];
        \draw[dashed, <->, lightgray] (-1,0,-1) -- (3,0,-1) node[midway, below] {$d$};
        \draw[dotted, thin, lightgray] (B1) -- ++(0,0,-1);
        \draw[dotted, thin, lightgray] (B2) -- ++(0,0,-1);
        
        \draw[dash dot, fill=none, red](0,0,0) circle (1);
        \draw[dash dot, fill=none, blue](0,0,0) circle (3);

        \draw[dashed, red] (B1) -- (0,0,0) node[midway] {$x_1$};
        \draw[dashed, blue] (0,0,0) -- (B2) node[midway] {$x_2$};
        
        \node (o) at (0,0,0)      [point=black, label=below:$c$];
        \node (r) at (1.25,2,1.25)    [point=purple, label=left:$S$];
        \draw[dashed, purple] (0,0,0) -- (0, 2, 0) node[midway, right]  {$y$} -- (1.25, 2, 0) node[midway, above]  {$x$} -- (r) node[midway, right] {$z$};

        \draw[->, gray] (0,0,0) -- (0.75,0,0) node[near end, below right] {$\vh{x}$};
        \draw[->, gray] (0,0,0) -- (0,0.75,0) node[near end, above] {$\vh{y}$};
        \draw[->, gray] (0,0,0) -- (0,0,0.75) node[near end, above left] {$\vh{z}$};

        \draw[dashed, gray] (0,0,0) -- (0,0,2);
        \draw[->, gray] (0.5,0,2) arc (0:270:0.5) node [midway, above] {$\Omega$};
    \end{tikzpicture}

    \caption{Geometry}
\end{figure}

The purple satellite is located by $\vv{r}=x\vh{x}+y\vh{y}+z\vh{z}$, and the spheres are the celestial bodies. The $xyz$ frame is not inertially stationary. Instead, they rotate with the larger bodies at a rate $\Omega$. The origin $c$ is the center of mass of $m_1$ and $m_2$ (and therefore inertially fixed). The $xy$ plane defines the plane in which $m_1$ and $m_2$ orbit, so $\vh{z}$ is defined by the direction of their angular momenta- the constant rotation at a rate $\Omega$ is a consequence of this. Note that $\vv{r}$ is not constrained to the $xy$ plane, but $m_1$ and $m_2$ definitionally are. 

\section*{Useful Relations}

We can also find a relationship between $x_1$, $x_2$, $d$, $m_1$, and $m_2$. Because $c$ is the barycenter,
\[x_1m_1=x_2m_2\]
or
\[\frac{x_1}{m_2}=\frac{x_2}{m_1}\]
We can use this to find that
\[\begin{aligned}
\frac{x_1}{x_1+x_2}&=\frac{x_1/m_2}{x_1/m_2+x_2/m_2}\\
&=\frac{x_2/m_1}{x_2/m_1+x_2/m_2}\\
&=\frac{1/m_1}{1/m_1+1/m_2}\\
&=\frac{m_2}{m_2+m_1}\\
\end{aligned}\]

Defining $d=x_1+x_2$ as the distance between the two celestial bodies, and $m=m_1+m_2$ as their total mass, we get that
\[\boxed{\frac{x_1}{d}=\frac{m_2}{m}}\]
and similarly
\[\boxed{\frac{x_2}{d}=\frac{m_1}{m}}\]


Which can be rewritten as
\[\boxed{\frac{x_1}{m_2}=\frac{x_2}{m_1}=\frac{d}{m}}\]

Lastly, we will define two additional vectors $\vv{r_1}$ and $\vv{r_2}$ which point from the first and second body respectively to the satellite.
\[\vv{r_1}=\left(x+x_1\right)\vh{x}+y\vh{y}+z\vh{z}\]
and
\[\vv{r_2}=\left(x-x_2\right)\vh{x}+y\vh{y}+z\vh{z}\]

Now we will solve for $\Omega$.
\[\begin{aligned}
    \vv{F_{\text{on2}}}&=-\frac{Gm_1m_2}{d^2}\vh{x}\\
    \cancel{m_2}\vv{a}_2^f&=-\frac{Gm_1\cancel{m_2}}{d^2}\vh{x}\\
    \dd{^f}{t}\left(\dd{^f}{t}x_2\vh{x}\right)&=-\frac{Gm_1}{d^2}\vh{x}\\
    \dd{^f}{t}\left(\cancelto{0}{\dd{^c x_2\vh{x}}{t}}+\Omega\vh{z}\times x_2\vh{x}\right)&=-\frac{Gm_1}{d^2}\vh{x}\\
    \dd{^f}{t}\Omega x_2\vh{y}&=-\frac{Gm_1}{d^2}\vh{x}\\
    \cancelto{0}{\dd{^c}{t}\Omega x_2 \vh{y}}+\Omega\vh{z}\times\Omega x_2\vh{y}&=-\frac{Gm_1}{d^2}\vh{x}\\
    -\Omega^2 x_2\vh{x}&=-\frac{Gm_1}{d^2}\vh{x}\\
    \Omega^2x_2&=\frac{Gm_1}{d^2}\\
    \Omega&=\sqrt{\frac{G}{d^2}\frac{m_1}{x_2}}\\
    \Omega&=\sqrt{\frac{Gm}{d^3}}\\
\end{aligned}\]

Defining $\mu$ conventionally as $\mu=Gm$,
\[\boxed{\Omega=\sqrt{\frac{\mu}{d^3}}}\]

\section*{Kinematics}
The transport theorem states that the inertial (fixed $f$ frame) derivative of a vector $u$ (expressed in the rotating $c$ frame) is

\[\dd{^f \vv{u}}{t}=\dd{^c\vv{u}}{t}+\vv{\omega}^{cf}\times\vv{u}\]

Where $\dd{^f}{t}$ denotes the derivative in the coordinates of the fixed frame $f$, and $\dd{^c}{t}$ denotes derivative in the coordinates of the rotating frame $c$, and $\vv{\omega}^{cf}$ denotes the angular velocity of $c$ in $f$. For this case, $\vv{\omega}^{cf}=\Omega\vh{z}$. We can find $\Omega$

For the satellite's position in the CR3BP frame $\vv{r}=x\vh{x}+y\vh{y}+z\vh{z}$, we will find the inertial acceleration to generate equations of motion.

\[
\begin{aligned}
    \vv{\dot{r}}&=\dd{^c\vv{r}}{t}+\vv{\omega}^{cf}\times\vv{r}\\
    &=\dd{^c}{t}\left(x\vh{x}+y\vh{y}+z\vh{z}\right)+\left(\Omega\vh{z}\times\left(x\vh{x}+y\vh{y}+z\vh{z}\right)\right)\\
    &=\left(\dot{x}\vh{x}+\dot{y}\vh{y}+\dot{z}\vh{z}\right)+\left(\Omega x\vh{y}-\Omega y\vh{x}\right)\\
    &=\left(\dot{x}-\Omega y\right)\vh{x}+\left(\dot{y}+\Omega x\right)\vh{y}+\dot{z}\vh{z}
\end{aligned}
\]

\[
\begin{aligned}
    \vv{\ddot{r}}&=\dot{\vv{r}}\\
    &=\dd{^c}{t}\left(\left(\dot{x}-\Omega y\right)\vh{x}+\left(\dot{y}+\Omega x\right)\vh{y}+\dot{z}\vh{z}\right)\\
        &\phantom{=}\quad+\Omega \vh{z}\times\left(\left(\dot{x}-\Omega y\right)\vh{x}+\left(\dot{y}+\Omega x\right)\vh{y}+\dot{z}\vh{z}\right)\\
    &=\left(\ddot{x}-\Omega \dot{y}\right)\vh{x}+\left(\ddot{y}+\Omega \dot{x}\right)\vh{y}+\ddot{z}\vh{z}\\
        &\phantom{=}\quad+\left(\left(\Omega\dot{x}-\Omega^2 y\right)\vh{y}-\left(\Omega\dot{y}+\Omega^2 x\right)\vh{x}\right)\\
    &=\left(\ddot{x}-2\Omega\dot{y}-\Omega^2 x\right)\vh{x} + \left(\ddot{y}+2\Omega\dot{x}-\Omega^2 y\right)\vh{y}+\ddot{z}\vh{z}\\
\end{aligned}
\]

\[\boxed{\vv{\ddot{r}}=\left(\ddot{x}-2\Omega\dot{y}-\Omega^2 x\right)\vh{x} + \left(\ddot{y}+2\Omega\dot{x}-\Omega^2 y\right)\vh{y}+\ddot{z}\vh{z}}\]

\section*{Equations of Motion}
We can now generate the equations of motion

\[\begin{aligned}
\sum_i \vv{F_i}&=m\vv{\ddot{r}}\\
\vv{F_1}+\vv{F_1}&=m\vv{\ddot{r}}\\
-\frac{\mu_1m}{r_1^3}\vv{r_1}-\frac{\mu_2m}{r_2^3}\vv{r_2}&=m\vv{\ddot{r}}\\
-\frac{\mu_1}{r_1^3}\vv{r_1}-\frac{\mu_2}{r_2^3}\vv{r_2}&=\left(\ddot{x}-2\Omega\dot{y}-\Omega^2 x\right)\vh{x} + \left(\ddot{y}+2\Omega\dot{x}-\Omega^2 y\right)\vh{y}+\ddot{z}\vh{z}
\end{aligned}\]

We now write this as three equations, one each in $x$, $y$, and $z$
\[\begin{aligned}
    -\frac{\mu_1}{r_1^3}(x+x_1)-\frac{\mu_2}{r_2^3}(x-x_2)&=\ddot{x}-2\Omega\dot{y}-\Omega^2 x\\
    -\frac{\mu_1}{r_1^3}y-\frac{\mu_2}{r_2^3}y&=\ddot{y}+2\Omega\dot{x}-\Omega^2 y\\
    -\frac{\mu_1}{r_1^3}z-\frac{\mu_2}{r_2^3}z&=\ddot{z}
\end{aligned}\]

Isolating the second derivatives,

\[\begin{aligned}
    \ddot{x}&=-\frac{\mu_1}{r_1^3}(x+x_1)-\frac{\mu_2}{r_2^3}(x-x_2)+2\Omega\dot{y}+\Omega^2 x\\
    \ddot{y}&=-\frac{\mu_1}{r_1^3}y-\frac{\mu_2}{r_2^3}y-2\Omega\dot{x}+\Omega^2 y\\
    \ddot{z}&=-\frac{\mu_1}{r_1^3}z-\frac{\mu_2}{r_2^3}z
\end{aligned}\]
    
We can now substitute $\Omega=\sqrt{\frac{\mu}{d^3}}=\sqrt{\frac{\mu_1+\mu_2}{d^3}}$

\[\boxed{\begin{aligned}
    \ddot{x}&=-\frac{\mu_1}{r_1^3}(x+x_1)-\frac{\mu_2}{r_2^3}(x-x_2)+2\sqrt{\frac{\mu}{d^3}}\dot{y}+\frac{\mu}{d^3}x\\
    \ddot{y}&=-\frac{\mu_1}{r_1^3}y-\frac{\mu_2}{r_2^3}y-2\sqrt{\frac{\mu}{d^3}}\dot{x}+\frac{\mu}{d^3}y\\
    \ddot{z}&=-\frac{\mu_1}{r_1^3}z-\frac{\mu_2}{r_2^3}z
\end{aligned}}\]

\section*{Nondimensional Equations of Motion}
We will now begin to non-dimensionalize the EOMs.

First, we will change the time unit. We will pick $t^\star=t\sqrt{\mu/d^3}$ (recall that $\sqrt{\mu/d^3}=\Omega$, which has dimensions of $\text{time}^{-1}$). Everywhere that a derivative is present, it is implied to be with respect to the time unit of 1 second. We must therefore switch from implied $\dd{}{t}$ to implied $\dd{}{t^\star}$.

\[\begin{aligned}
    \dd{(\phantom{x})}{t}&=\dd{(\phantom{x})}{t^\star}\dd{t^\star}{t}\\
    &=\dd{(\phantom{x})}{t^\star}\sqrt{\frac{\mu}{d^3}}
\end{aligned}\]

From this, it can be seen that to make a derivative with time implied to be with the nondimensional time, $\sqrt{d^3/\mu}$ must be multiplied for each derivative taken. The EOMs can now be rewritten this way, with dots implied to be relative to the nondimensional time unit

\[\begin{aligned}
    \left(\dd{^2x}{t^{\star2}}\frac{\mu}{d^3}\right)&=-\frac{\mu_1}{r_1^3}(x+x_1)-\frac{\mu_2}{r_2^3}(x-x_2)+2\sqrt{\frac{\mu}{d^3}}\left(\dd{y}{t^\star}\sqrt{\frac{\mu}{d^3}}\right)+\frac{\mu}{d^3}x\\
    \left(\dd{^2y}{t^{\star2}}\frac{\mu}{d^3}\right)&=-\frac{\mu_1}{r_1^3}y-\frac{\mu_2}{r_2^3}y-2\sqrt{\frac{\mu}{d^3}}\left(\dd{x}{t^\star}\sqrt{\frac{\mu}{d^3}}\right)+\frac{\mu}{d^3}y\\
    \left(\ddot{z}\frac{\mu}{d^3}\right)&=-\frac{\mu_1}{r_1^3}z-\frac{\mu_2}{r_2^3}z
\end{aligned}\]

Some algebraic simplifications can now be made

\[\begin{aligned}
    \dd{^2x}{t^{\star2}}&=-\frac{m_1 d^3}{m r_1^3}(x+x_1)-\frac{m_2 d^3}{m r_2^3}(x-x_2)+2\dd{y}{t^\star}+x\\
    \dd{^2y}{t^{\star2}}&=-\frac{m_1 d^3}{m r_1^3}y-\frac{m_2 d^3}{m r_2^3}y-2\dd{x}{t^\star}+y\\
    \dd{^2z}{t^{\star2}}&=-\frac{m_1 d^3}{m r_1^3}z-\frac{m_2 d^3}{m r_2^3}z
\end{aligned}\]

Next, we can define nondimensional distances and masses. The relationship between a nondimensional distance $L^\star$ and its dimensional counterpart is $L^\star=L/d$. Furthermore, we define nondimensional masses a similar relatonship $M^\star=M/m$. With this defined, we can make some substitutions in the EOMs.

\[\begin{aligned}
    \dd{^2x}{t^{\star2}}&=-\frac{m_1^\star}{r_1^{\star3}}(x+x_1)-\frac{m_2^\star}{r_2^{\star3}}(x-x_2)+2\dd{y}{t^\star}+x\\
    \dd{^2y}{t^{\star2}}&=-\frac{m_1^\star}{r_1^{\star3}}y-\frac{m_2^\star}{r_2^{\star3}}y-2\dd{x}{t^\star}+y\\
    \dd{^2z}{t^{\star2}}&=-\frac{m_1^\star}{r_1^{\star3}}z-\frac{m_2^\star}{r_2^{\star3}}z
\end{aligned}\]

By dividing both sides by $d$, the remaining distances can be made nondimensional.

\[\begin{aligned}
    \dd{^2x^\star}{t^{\star2}}&=-\frac{m_1^\star}{r_1^{\star3}}(x^\star+x_1^\star)-\frac{m_2^\star}{r_2^{\star3}}(x^\star-x_2^\star)+2\dd{y^\star}{t^\star}+x^\star\\
    \dd{^2y^\star}{t^{\star2}}&=-\frac{m_1^\star}{r_1^{\star3}}y^\star-\frac{m_2^\star}{r_2^{\star3}}y^\star-2\dd{x^\star}{t^\star}+y^\star\\
    \dd{^2z^\star}{t^{\star2}}&=-\frac{m_1^\star}{r_1^{\star3}}z^\star-\frac{m_2^\star}{r_2^{\star3}}z^\star
\end{aligned}\]

The final set of substitutions will now be made: $m_2=m-m_1 \to m_2^\star=1-m_1^\star$. Recall from the useful relations that $\frac{x_1}{d}=\frac{m_2}{m}$, Therefore $x_1^\star=1-m_1^\star$ (and $x_2^\star=m_1^\star$). While $r_1^\star$ and $r_2^\star$ can be written in terms of $x_1^\star$, $x^\star$, $y^\star$, and $z^\star$, I will not do this as it makes the equations much less compact.

\[\begin{aligned}
    \dd{^2x^\star}{t^{\star2}}&=-\frac{m_1^\star}{r_1^{\star3}}(x^\star+1-m_1^\star)-\frac{1-m_1^\star}{r_2^{\star3}}(x^\star-m_1^\star)+2\dd{y^\star}{t^\star}+x^\star\\
    \dd{^2y^\star}{t^{\star2}}&=-\frac{m_1^\star}{r_1^{\star3}}y^\star-\frac{1-m_1^\star}{r_2^{\star3}}y^\star-2\dd{x^\star}{t^\star}+y^\star\\
    \dd{^2z^\star}{t^{\star2}}&=-\frac{m_1^\star}{r_1^{\star3}}z^\star-\frac{1-m_1^\star}{r_2^{\star3}}z^\star
\end{aligned}\]

I will write this with the nondimensionalization implicit instead of explicit. In this space, all parameters represent their nondimensional counterpart.

\[\boxed{\begin{aligned}
    \ddot{x}&=-\frac{m_1}{r_1^3}(x+1-m_1)-\frac{1-m_1}{r_2^3}(x-m_1)+2\dot{y}+x\\
    \ddot{y}&=-\frac{m_1}{r_1^3}y-\frac{1-m_1}{r_2^3}y-2\dot{x}+y\\
    \ddot{z}&=-\frac{m_1}{r_1^3}z-\frac{1-m_1}{r_2^3}z
\end{aligned}}\]

If instead $m_2$ is used as the characterizing parameter (rather than $m_1$), the EOMs are

\[\boxed{\begin{aligned}
    \ddot{x}&=-\frac{1-m_2}{r_1^3}(x+m_2)-\frac{m_2}{r_2^3}(x-1+m_2)+2\dot{y}+x\\
    \ddot{y}&=-\frac{1-m_2}{r_1^3}y-\frac{m_2}{r_2^3}y-2\dot{x}+y\\
    \ddot{z}&=-\frac{1-m_2}{r_1^3}z-\frac{m_2}{r_2^3}z
\end{aligned}}\]

The solutions to this differential equation solve the non-dimensional CR3BP, in which the bodies' masses add to 1 and the distance between them is 1. To translate it into real solutions, the time scale, distance scale, and mass scale must be applied. Therefore the only thing in this equation that depends upon the speciifc two-body system of interest is $m_1$, or the fraction of the primary mass to the total masses of both planetary .

\section*{Energy Analysis}

There is a "canonical" way of writing 

\end{document}