\documentclass{article}
\usepackage{amsmath}
\usepackage{tikz}
\usepackage{float}


\pagestyle{empty}
\begin{document}

First, we will draw the scenario
\begin{figure}[H]
    \centering
    \begin{tikzpicture}[
        planet/.style = {circle, draw=black, ball color=#1, node contents={}, inner sep = 0.1cm}, 
        point/.style = {circle, draw=#1, fill=#1, node contents={}, inner sep=0.05cm},
        >=latex
        ]

        \node (A) at (-1,0)     [planet=green, label=left:$B_1$];
        \node (B) at (5,0)      [planet=blue, label=right:$B_2$];
        \node (o) at (0,0)      [point=black, label=right:$c$];
        \node (r) at (3.5,2)    [point=red, label=left:$S$];
        
    \end{tikzpicture}

    \caption{Geometry. $S$ is the satellite, $B$ are the celestial bodies}
\end{figure}

\begin{figure}[H]
    \centering
    \begin{tikzpicture}[
        point/.style = {circle, draw=#1, fill=#1, node contents={}, inner sep=0.05cm},
        >=latex
        ]

        \node (B1) at (-1,0)     [point=green, label=left:$B_1$];
        \node (B2) at (5,0)      [point=blue, label=right:$B_2$];
        \node (o) at (0,0)      [point=black, label=below:$c$];
        \node (S) at (3.5,2)    [point=red, label=above:$S$];

        \draw[->, color=blue] (B2) -> node[below] {$\vec{r}_2$} (S);
        \draw[->, color=green] (B1) -> node[above] {$\vec{r}_1$} (S);
        \draw[->, color=black] (o) -> node[below] {$\vec{r}$} (S);
    \end{tikzpicture}
    \caption{Vectors. Note $d_1$ refers to the distance from $c$ to $B_1$ and $d_2$ refers to the distance from $c$ to $B_2$. Predictably, $d_{12}=d_1+d_2$ is the distance between the bodies}
\end{figure}



\end{document}
